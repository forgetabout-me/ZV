\def\PAPER{a4paper} % papírméret
\def\FONTSIZE{10pt} % betűméret
\def\OPTIONS{twoside} % oldalbeállítások
\def\TARSSZERZO{} % Ha tettél hozzá érdemi munkát, és közreadnád, írd fel magad

\documentclass[magyar,\PAPER,\OPTIONS]{article}
\usepackage{imakeidx}    %Tárgymutatóhoz (index)
\usepackage[utf8]{inputenc}
\usepackage[T1]{fontenc}
\usepackage{mathtools}
\usepackage{amsthm}
\usepackage{enumitem}
\usepackage[magyar]{babel}
\usepackage[sharp]{easylist}
\usepackage[
top=2cm,
bottom=2cm,
left=2cm,
right=1cm
]{geometry}
\usepackage[]{hyperref}
%\usepackage{showframe}
%\usepackage{titlesec}

%\titleformat{\section}[display]{\bfseries}{\thesection. tétel:}{0pt}{}[]
\let\OldEasylist\easylist
\let\OldEndEasylist\endeasylist
\renewenvironment{easylist}{
	\OldEasylist
	\ListProperties(Numbers2=l,FinalMark2={)},Hide2=1,Progressive*=3ex, Start1=1)
}{
	\OldEndEasylist
}
\newcounter{descriptcount}
\newlist{enumdescript}{description}{2}
\setlist[enumdescript,1]{
	before={\setcounter{descriptcount}{0}
		\renewcommand*\thedescriptcount{\arabic{descriptcount}}}
	,font=\bfseries\stepcounter{descriptcount}\thedescriptcount.~
}
\setlist[enumdescript,2]{
	before={\setcounter{descriptcount}{0}
		\renewcommand*\thedescriptcount{\roman{descriptcount}}}
	,font=\bfseries\stepcounter{descriptcount}\thedescriptcount.~
}

\newtheorem*{definition}{Definíció}
\newtheorem*{theorem}{Tétel}
\newtheorem*{note}{Megjegyzés}

\makeindex

\title{Záróvizsga tételsor mérnökinformatikus hallgatóknak\thanks{A Debreceni Egyetem mérnökinformatikus alapszakhoz}}
\author{Palkovics Dénes, \TARSSZERZO}
\date{2019}


\setlength{\parskip}{6pt plus 3pt minus 3pt} % Bekezdések térközölése
\setcounter{tocdepth}{3} % Tartalomjegyzék mélysége. Alapegyég 3

\begin{document}
\maketitle
%\cleardoublepage
%\begin{titlepage}
\begin{center}
\EGYETEM \\
\KAR \\
%\TANSZEK
\end{center}

\vfill

\begin{center}
\LARGE
\textbf{\CIM}
\normalsize
\end{center}

\vfill

\begin{minipage}[rt]{\linewidth}
\centering
\textit{Készítette:}\\
\textbf{\SZERZO}\\
\SZAK
\end{minipage}

\vfill

\begin{center}
Debrecen, \VEDESEVE
\end{center}

\end{titlepage}

\textbf{ A záróvizsga tematikája és tartalma}\\
A záróvizsgán kettő kérdésre kell válaszolni, egyre az általános kérdések közül, egyre pedig a specializációnak megfelelő kérdések közül
\tableofcontents
\cleardoublepage
\section{Általános kérdések}
%-------------------------------------------------------------------------------
\subsection{Az informatika logikai alapjai}
%-------------------------------------------------------------------------------
\def\InterpretOnNu{^{\langle U, \rho \rangle}_{\nu}}
\subsubsection{Az elsőrendű matematikai logikai nyelv.}
\begin{definition}[Elsőrendű nyelv]
	Klasszikus elsőrendű nyelven az $$ L^{(1)} = \langle LC,Var,Con,Term,Form\rangle $$ rendezett ötöst értjük, ahol\\
	\begin{easylist}
	# $LC = \{\neg,\supset,\land,\lor,\equiv,=,\forall,\exists,(,)\}$ a nyelv logikai konstansainak halmaza\footnote{A logikai konstansok olyan nyelvi eszközök, amelyek jelentését a szemantikai szabályok (logikai kalkulusok esetén az axiómák) rögzítik. Egy adott logikai rendszer esetén a logikai konstansok rögzített jelentéssel (rögzített szemantikai értékkel)rendelkeznek, jelentésük (szemantikai értékük) minden interpretációban megegyezik. Egy adott logikai rendszer esetén a logikai konstansokat általában az adott logikai rendszer nyelvének $LC$	halmaza tartalmazza.}
	# $Var = \{x_{n}| n = 0,1,2,\dots\}$ a nyelv változóinak megszámlálhatóan végtelen halmaza\footnote{A köznyelvi mondatokban nevek helyett néha névmásokkal utalunk egyes individuumokra (objektumokra). A tudományos nyelvben gyakran kívánatos analóg kifejezési formák megadása. A szabatosság, az egyértelműség és a tömörség érdekében ilyenkor mesterséges névmásokat vezetnek be, amelyeket változóknak neveznek.}
	# $Con = \bigcup_{n=0}^\infty(\mathcal{F}(n)\cup\mathcal{P}(n))$ a nyelv nemlogikai konstansainak legfeljebb megszámlálhatóan végtelen halmaza\footnote{A nemlogikai konstansok, más néven paraméterek olyan nyelvi eszközök, amelyek jelentését az interpretáció rögzíti. Egy adott logikai rendszer esetén a nemlogikai konstansok (a paraméterek) nem rendelkeznek rögzített jelentéssel (rögzített szemantikai értékkel), jelentésük (szemantikai értékük) interpretációról interpretációra változhat. Egy adott logikai rendszer esetén a nemlogikai konstansokat általában az adott logikai rendszer nyelvének $Con$ halmaza tartalmazza.}
	## $\mathcal{F}(0)$ a névparaméterek (névkonstansok),
	## $\mathcal{F}(n)$ az $ n $ argumentumú függvényjelek (műveleti jelek),
	## $\mathcal{P}(0)$ a állításparaméterek (állításkonstansok),
	## $\mathcal{P}(n)$ az $ n $ argumentumú predikátumparaméterek (predikátumkonstansok) halmaza.
	# Az $LC,Var,\mathcal{F}(n),\mathcal{P}(n)$ halmazok ($n = 0,1,2,\dots$) páronként diszjunktak.
	# A nyelv terminusainak a halmazát, azaz a $Term$ halmazt az alábbi induktív definíció adja:
	## $Var \cup \mathcal{F}(0)\subseteq Term$
	## Ha $f\in\mathcal{F}(n), (n=1,2,\dots)$,és $t_1,t_2,\dots,t_n \in Term$, akkor $f(t_1,t_2,\dots,t_n)\in Term$
	# \label{itm:induction}A nyelv formuláinak halmazát, azaz a $Form$ halmazt az alábbi induktív definíció adja meg:
	## \label{itm:rule1} $\mathcal{P}\subseteq Form$
	## \label{itm:rule2} Ha $t_1,t_2 \in Term$, akkor $(t_1 = t_2) \in Form$
	## \label{itm:rule3} Ha $P \in \mathcal{P}, (n=1,2,\dots)$, és $t_1,t_2,\dots,t_n\in Term$, akkor $P(t_1,t_2,\dots,t_n)\in Form$
	## Ha $A \in Form$, akkor $\neg A \in Form$
	## Ha $A,b \in Form$, akkor $(A \supset B),(A \land B),(A\lor B),(A\equiv B) \in Form$
	## Ha $x\in Var, A\in Form$, akkor $\forall x A, \exists x A \in Form$
	\end{easylist}
\end{definition}
\begin{note}
	Azokat a formulákat, amelyek a \ref{itm:induction} \ref{itm:rule1}, \ref{itm:rule2}, \ref{itm:rule3} szabályok által jönnek létre, atomi formuláknak\index{atomi formula} vagy prímformuláknak\index{prímforumla} nevezzük.
\end{note}

\subsubsection{A nyelv interpretációja, formulák igazságértéke az interpretációban adott változókiértékelés mellett.}
\begin{definition}[interpretáció (elsőrendű)]
	Az $\langle U, \rho\rangle$ párt az $L^{(1)}$ nyelv egy interpretációjának nevezzük, ha\\
	\begin{easylist}
		# $U \neq \emptyset$ azaz $U$ nemüres halmaz
		# $Dom(\rho) = Con$ azaz a $\rho$ a $Con$ halmazon értelmezett függvény, amelyre teljesülnek a következők:
		## Ha $a \in F(0)$, akkor $\rho(a) \in U$
		## Ha $f \in \mathcal{F}(n)$ ahol $n\neq 0$, akkor $\rho(f)$ az $U^{(n)}$ halmazon értelmezett az $U$ halmazba képező függvény ($\rho(f) : U^{(n)} \rightarrow U $)
		## Ha $p \in \mathcal{P}(0)$, akkor $\rho(p) \in {0, 1}$
		## Ha $P \in \mathcal{P}(n)$ ahol $n \neq 0$, akkor $\rho(P) \subseteq U^{(n)}$
	\end{easylist}
\end{definition}
\begin{definition}[értékelés (elsőrendű)]
	Legyen $L^{(1)} = \langle LC, Var, Con, Term, Form\rangle$ egy elsőrendű nyelv, $\langle U, \rho\rangle$ pedig a nyelv egy interpretációja. Az $\langle U, \rho\rangle$ interpretációra támaszkodó $\nu$ értékelésen egy olyan függvényt értünk, amely teljesíti a következőket:
	\begin{itemize}
		\item  $ Dom(\nu) = Var $
		\item  $ Ha x \in Var$, akkor $\nu(x) \in U $
	\end{itemize}
\end{definition}
\begin{definition}[értékelés (elsőrendű)]
	Legyen $L^(1) = (LC, Var, Con, Term, Form)$ egy elsőrendű nyelv, $\langle U, \rho \rangle$ pedig a nyelv egy interpretációja, $\nu$ pedig az $\langle U, \rho \rangle$ interpretációra támaszkodó értékelés.\\
	\begin{easylist}
		# Ha $a \in F(0)$, akkor 
				$|a|\InterpretOnNu = \rho(a)$
		# Ha $x \in Var$, akkor 
				$|x|^{\langle U,\rho \rangle}_{\nu} = \nu(x)$
		# Ha $f \in F(n)$,$(n = 1,2,\dots)$ és $t_1,t_2,\dots,t_n \in Term$, akkor 
				$$|f(t_1,t_2,\dots,t_n)|\InterpretOnNu = \rho(f)(|t_1|\InterpretOnNu,|t_2|\InterpretOnNu,\dots,|t_n|\InterpretOnNu)$$
		# Ha $p \in P(0)$, akkor
				 $|p|_\nu^{\langle U,\rho\rangle} = \rho(p)$
		# Ha $t_1, t_2 \in Term$, akkor
		\begin{equation}
			|(t_1 = t_2)|\InterpretOnNu =
			\begin{cases}
				1, & \text{ha $|t_1|\InterpretOnNu = |t_2|\InterpretOnNu$}\\
				0, & \text{egyébként.}
			\end{cases}
		\end{equation}
		# Ha $P \in P(n) ahol n = 0, t1 , \dots , tn \in Term$, akkor
		\begin{equation}
			|P(t_1,t_2,\dots,t_n)|\InterpretOnNu =
			\begin{cases}
				1, & \text{ha $\big(|t_1|\InterpretOnNu,|t_2|\InterpretOnNu,\dots,|t_n|\InterpretOnNu\big)\in \rho(P)$}\\
				0, & \text{egyébként.}
			\end{cases}
		\end{equation}
		# Ha $A \in Form$, akkor
			$|\neg A|\InterpretOnNu = 1 - |A|\InterpretOnNu$.
		# Ha $A,B \in Form$, akkor
			\begin{equation}
				|(A\supset B)|\InterpretOnNu
				\begin{cases}
				0, & \text{ha $|A|\InterpretOnNu = 1$, és $|B|\InterpretOnNu = 0$}\\
				1, & \text{egyébként.}
				\end{cases}
			\end{equation}
			
			\begin{equation}
				|(A\land B)|\InterpretOnNu
				\begin{cases}
				1, & \text{ha $|A|\InterpretOnNu = 1$, és $|B|\InterpretOnNu = 1$}\\
				0, & \text{egyébként.}
				\end{cases}
			\end{equation}
			
			\begin{equation}
				|(A\lor B)|\InterpretOnNu
				\begin{cases}
				0, & \text{ha $|A|\InterpretOnNu = 0$, és $|B|\InterpretOnNu = 0$}\\
				1, & \text{egyébként.}
				\end{cases}
			\end{equation}
			
			\begin{equation}
				|(A\equiv B)|\InterpretOnNu
				\begin{cases}
				1, & \text{ha $|A|\InterpretOnNu = |B|\InterpretOnNu$}\\
				0, & \text{egyébként.}
				\end{cases}
			\end{equation}
		# Ha $A \in Form, x \in Var$, akkor
			\begin{equation}
				|(\forall_x A)|\InterpretOnNu =
				\begin{cases}
				0, & \text{ha van olyan $u \in U$, hogy$|A|^{\langle U, \rho \rangle}_{\nu [x:u]} = 0$}\\
				1, & \text{egyébként.}
				\end{cases}
			\end{equation}
			
			\begin{equation}
				|(\exists_x A)|\InterpretOnNu =
				\begin{cases}
				1, & \text{ha van olyan $u \in U$, hogy$|A|^{\langle U, \rho \rangle}_{\nu [x:u]} = 1$}\\
				0, & \text{egyébként.}
				\end{cases}
			\end{equation}
	\end{easylist}
\end{definition}

\subsubsection{Logikai törvény, logikai következmény.}
\begin{definition}[modell]
	Legyen $L^{(1)} = (LC, Var, Con, Term, Form)$ egy elsőrendű nyelv és $\Gamma \subseteq Form$ egy tetszőleges formulahalmaz. Az $(U, \rho, \nu)$ rendezett hármas elsőrendű modellje a $\Gamma$ formulahalmaznak, ha 
	\begin{itemize}
		\item  $(U, \rho)$ egy interpretációja az $L^{(1)}$ nyelvnek; 
		\item  $\nu$ egy $(U, \rho)$ interpretációra támaszkodó értékelés; 
		\item  minden  $A \in \Gamma$ esetén $|A|\InterpretOnNu = 1$.
	\end{itemize}
	
\end{definition}
\begin{definition}
	Legyen $L^{(1)} = (LC, Var, Con, Term, Form)$ egy elsőrendű nyelv és $\Gamma \subseteq Form$ egy tetszőleges formulahalmaz, $A,B \in Form$ egy tetszőleges formulák.
	\begin{itemize}
		\item Egy $\Gamma$ formulahalmaz \emph{kielégíthető}, ha van (elsőrendű) modellje;
		\item Egy $\Gamma$ formulahalmaz \emph{kielégíthetetlen}, ha nem kielégíthető, azaz nincs modellje;
		\item Az $A$ formula \emph{modellje} az $\{A\}$ egyelemű formulahalmaz modelljét értjük;
		\item Az $A$ formula \emph{kielégíthető}, ha $\{A\}$ formulahalmaz kielégíthető;
		\item Az $A$ formula \emph{kielégíthetetlen}, ha $\{A\}$ formulahalmaz kielégíthetetlen;
		\item A  $\Gamma$ formulahalmaznak \underline{logikai következménye} az $A$ formula, ha a $\Gamma \cup \{\neq A\}$ formulahalmaz kielégíthetetlen. Jelölés: $\Gamma \models A$ 
		\item Az $A$ formulának \underline{logikai következménye} a $B$ formula, ha a $\{A\} \models B$. Jelölés: $A \models B$ 
		\item Az $A$ formula \emph{érvényes} (\underline{logikai törvény})\label{def:logikai törvény}, ha $\emptyset \models A$, azaz ha az $A$ formula \underline{logikai következménye} az üres halmaznak. Másképpen, ha minden $ \langle U, \rho \rangle$ interpretációjában, minden $\nu$ értékelés szempontjából $|A|\InterpretOnNu = 1$ Jelölés: $\models A$
		\item Az $A$ és a $B$ formula \emph{logikailag ekvivalens}, ha $A \models B$ és $B \models A$. Jelölés: $A \Leftrightarrow B$ 
	\end{itemize}
\end{definition}

\subsubsection{Logikai ekvivalencia, normálformák.}
\begin{definition}[Logikai ekvivalencia]
	lásd:a \ref{def:logikai törvény} fejezet definíciója.
\end{definition}
\begin{definition}[elemi konjunkció]
	Legyen $L^{(0)} = (LC, Con, Form)$ egy nulladrendű nyelv. Ha az $A \in Form$ formula literál vagy különböző alapú literálok konjunkciója, akkor $A$-t elemi konjunkciónak nevezzük. 
\end{definition}
\begin{definition}[elemi diszjunkció]
	Legyen $L^{(0)} = (LC, Con, Form)$ egy nulladrendű nyelv. Ha az $A \in Form$ formula literál vagy különböző alapú literálok diszjunkciója, akkor $A$-t elemi diszjunkciónak nevezzük. 
\end{definition}
\begin{definition}[diszjunktív normálforma]
	Egy elemi konjunkciót vagy elemi konjunkciók diszjunkcióját diszjunktív normálformának nevezzük. 
\end{definition}
\begin{definition}[konjunktív normálforma]
	Egy elemi diszjunkciót vagy elemi diszjunkciók konjunkcióját konjunktív normálformának nevezzük.
\end{definition}
\begin{definition}
	Legyen $L^{(0)} = (LC, Con, Form)$ egy nulladrendű nyelv és $A \in Form$ egy formula. Ekkor létezik olyan $B \in Form$, hogy
	\begin{itemize}
		\item $A\Leftrightarrow B$
		\item $B$ diszjunktív vagy konjunktív normálformájú. 
	\end{itemize}

\end{definition}

\subsubsection{Kalkulusok (Gentzen-kalkulus).}

\paragraph{Logikai kalkulus} 
Logikai kalkuluson olyan adott nyelv formuláihoz tartozó formális rendszert, szabályrendszert értünk, amely pusztán szintaktikailag, szemantika nélkül ad meg egy következményrelációt. A logikai kalkulus tehát egy axiómarendszer, amely magában a logikai tautológiákat állítja elő, adott formulákat ideiglenesen hozzávéve (premissza) pedig más formulákra (konklúzió) lehet jutni (következtetni) vele.

\paragraph{Gentzen-féle szekvenciakalkulus}
Ebben a kalkulusban nem formulákra vonatkoznak a szabályok és nem is formulák alkotják az axiómákat, hanem a formulák eddigi szerepét az ún. szekvencia töltik be. Szekvenciának nevezzük a
$$\Gamma \vdash \Delta$$
alakú jelsorozatokat, ahol $\Gamma$ és $\Delta$ olyan rendezett jelsorozatok, amelyeknek minden tagja egy formula.
\begin{definition}[axiómasémák]
	Legyen $L^{(0)} = (LC, Con, Form)$ egy nulladrendű nyelv (a klasszikus állításlogika nyelve). A nulladrendű kalkulus (klasszikus állításkalkulus) axiómasémái (alapsémái):
	\begin{easylist}
		# $A \supset (B \supset A)$
		# $(A \supset (B \supset C)) \supset ((A \supset B) \supset (A \supset C))$
		# $(\neg A \supset \neg B) \supset (B \supset A)$
	\end{easylist}
\end{definition}

Az axiómaséma szabályos behelyettesítésén olyan formulát értünk, amely az axiómasémából a benne szereplő betűk tetszőleges formulával való helyettesítése útján jön létre. A nulladrendű kalkulus (klasszikus állításkalkulus) axiómái az axiómasémák szabályos behelyettesítései. 

\begin{definition}[szintaktikai következmény]
	Legyen $L^{(0)} = (LC, Con, Form)$ egy nulladrendű nyelv, $\Gamma \subseteq Form$ egy tetszőleges formulahalmaz. A $\Gamma$ formulahalmaz szintaktikai következményeinek induktív definíciója:

	Bázis:
	\begin{itemize}
		\item Ha $A \in \Gamma$, akkor $\Gamma \vdash A$ 
		\item Ha $A$ axióma, akkor $\Gamma \vdash A$. 
	\end{itemize}
	
	Szabály (leválasztási szabály): 
	\begin{itemize}
		\item Ha $\Gamma \vdash B$, és $\Gamma \vdash (B \subset A)$, akkor $\Gamma \vdash A$. 
	\end{itemize}
\end{definition}

\begin{definition}[szintaktikai következmény]
Legyen $L^{(0)} = (LC, Con, Form)$ egy nulladrendű nyelv és $A, B \in Form$ két tetszőleges formula. Az $A$ formulának szintaktikai következménye a $B$ formula, ha $\{A\} \vdash B$. Jelölés: $A \vdash B$ 
\end{definition}

\begin{definition}[szekvencia]
Legyen $L^{(0)} = (LC, Con, Form)$ egy nulladrendű nyelv, $\Gamma \subseteq Form$ egy formulahalmaz és $A \in Form$ egy formula. Ha az $A$ formula szintaktikai következménye a $\Gamma$ formulahalmaznak, akkor a $\Gamma \vdash A$ jelsorozatot szekvenciának nevezzük. 
\end{definition}

\begin{definition}[levezethetőség]
Legyen $L^{(0)} = (LC, Con, Form)$ egy nulladrendű nyelv és $A \in Form$ egy tetszőleges formula. Az $A$ formula levezethető, ha $\emptyset \vdash A$, azaz ha az $A$ formula szintaktikai következménye az üres halmaznak. Jelölés: $\vdash A$ 
\end{definition}

\begin{definition}[természetes levezetés szabályai]
Legyen $L^{(0)} = (LC, Con, Form)$ egy nulladrendű nyelv  $\Gamma, \Delta \subseteq Form$ és $A, B, C \in Form$. A természetes levezetés által az $L^{(0)}$ nyelvben bizonyítható következményrelációk alábbiak:

Bázis:
\begin{equation}
	\frac{\omega}{\Gamma,A \vdash A}
\end{equation}
Szabályok:
\begin{itemize}
	\item Struktúrális szabályok:
	\begin{itemize}
		\item Bővítés $\frac{\Gamma \vdash A}{\Gamma, B \vdash A} $
		\item Felcserélés $\frac{\Gamma,B,C,\Delta\vdash A} {\Gamma,C,B,\Delta\vdash A} $
		\item Szűkítés $\frac{\Gamma,B,B,\Delta\vdash A}
		{\Gamma,B,\Delta\vdash A} $
		\item Metszet $\frac{\Gamma\vdash A \Delta,A\vdash B}{\Gamma,\Delta\vdash B} $
	\end{itemize}
	\item Logikai szabályok:
	\begin{itemize}
		\item Implikáció szabályai:
		\begin{itemize}
			\item bevezető: $\frac{\Gamma,A\vdash B}{\Gamma\vdash A \supset B} $
			\item alkalmazó: $\frac{\Gamma\vdash A \Gamma\vdash A \supset B}{\Gamma \vdash B} $
		\end{itemize}
		\item Negáció szabályai:
		\begin{itemize}
			\item bevezető:$\frac{\Gamma,A\vdash B \Gamma,A\vdash \neg B}{\Gamma\vdash \neg A} $
			\item alkalmazó:$\frac{\Gamma\vdash \neg\neg A}{\Gamma \vdash A}$
		\end{itemize}
		\item Konjunkció szabályai:
		\begin{itemize}
			\item bevezető:$\frac{\Gamma\vdash A \Gamma\vdash B}{\Gamma\vdash A\land B}$
			\item alkalmazó:$\frac{\Gamma,A,B\vdash C}{\Gamma,A\land B,\vdash C}$
		\end{itemize}
		\item Diszjunkció szabályai:
		\begin{itemize}
			\item bevezető:$\frac{\Gamma\vdash A}{\frac{\Gamma\vdash A \lor B \Gamma\vdash B}{\Gamma\vdash A \lor B}}$
			\item alkalmazó:$\frac{\Gamma,A\vdash C \Gamma,B\vdash C}{\Gamma,A\lor B\vdash C}$
		\end{itemize}
		\item (Materiális) ekvivalencia szabályai:
		\begin{itemize}
			\item bevezető:$\frac{\Gamma,A\vdash B \Gamma,B\vdash A}{\Gamma\vdash A\equiv B}$
			\item alkalmazó:$\frac{\Gamma\vdash A \Gamma\vdash A\equiv B}{\frac{\Gamma\vdash B \Gamma\vdash B \Gamma\vdash A \equiv B}{\Gamma\vdash A}}$
		\end{itemize}
	\end{itemize}
\end{itemize}
\end{definition}

%-------------------------------------------------------------------------------
%-------------------------------------------------------------------------------
\subsection{Operációs rendszerek}
%-------------------------------------------------------------------------------
\subsubsection{Operációs rendszerek fogalma, felépítése, osztályozásuk.}
\paragraph{Operációs rendszerek fogalma} 
Egy program, amely közvetítő szerepet játszik a számítógép felhasználója
és a számítógéphardver között.
Az operációs rendszer feladata, hogy a felhasználónak egy olyan egyenértékű kiterjesztett
vagy virtuális gépet nyújtson, amelyiket egyszerűbb programozni, mint a mögöttes hardvert
\paragraph{Operációs rendszerek felépítése}
Az operációs rendszerek alapvetően három részre bonthatók:
	\begin{itemize}
	\item a felhasználói felület (a shell, amely lehet egy grafikus felület, vagy egy szöveges)
	\item alacsony szintű segédprogramok
	\item kernel (mag), amely közvetlenül a hardverrel áll kapcsolatban.
	\end{itemize}
\paragraph{Operációs rendszerek osztályozása}
	\begin{enumerate}
	\item Az operációs rendszer alatti hardver "mérete" szerint:
		\begin{itemize}
		\item mikroszámítógépek operációs rendszerei
		\item kisszámítógépek, esetleg munkaállomások operációs rendszerei
		\item nagygépek (Main Frame Computers, Super Computers) operációs rendszerei
		\end{itemize}
	\item A kapcsolattartás típusa szerint:
		\begin{itemize}
		\item kötegelt feldolgozású operációs rendszerek vezérlőkártyás kapcsolattartással
		\item interaktív operációs rendszerek.
		\end{itemize}
	\item cél szerint: általános felhasználású vagy céloperációs rendszer
	\item a processzkezelés: single-tasking, multi-tasking
	\item a felhasználók száma szerint: single, multi
	\item CPU-idő kiosztása szerint: szekvenciális, megszakítás vezérelt, event-polling, time-sharing
	\item a memóriakezelés megoldása szerint: valós és virtuális címzésű
	\end{enumerate}

\subsubsection{Az operációs rendszerek jellemzése (komponensei és funkciói).}
\paragraph{Operációs rendszerek komponensei:}
\begin{description}
	\item[Eszközkezelők (Device Driver)] Felhasználók elől el fedik a perifériák különbségeit, egységes kezelői felületet kell biztosítani.
	\item[Megszakítás kezelés (Interrupt Handling)] Alkalmas perifériák felől érkező kiszolgálási igények fogadására, megfelelő ellátására.
	\item[Rendszerhívás, válasz (System Call, Reply)] az operációs rendszer magjának ki kell szolgálnia a felhasználói alkalmazások (programok) erőforrások iránti igényeit úgy, hogy azok lehetőleg észre se vegyék azt, hogy nem közvetlenül használják a perifériákat$\leftarrow$ programok által kiadott rendszerhívások, melyekre rendszermag válaszokat küldhet.
	\item[Erőforrás kezelés (Resource Management)] Az egyes eszközök közös használatából származó konfliktusokat meg kell előznie, vagy bekövetkezésük esetén fel kell oldania.
	\item[Processzor ütemezés (CPU Scheduling)] Az operációs rendszerek ütemező funkciójának a várakozó munkák között valamilyen stratégia alapján el kell osztani a processzor idejét, illetve vezérelnie kell a munkák közötti átkapcsolási folyamatot.
	\item[Memóriakezelés (Memory Management)] Gazdálkodnia kell a memóriával, fel kell osztania azt a munkák között úgy, hogy azok egymást se zavarhassák, és az operációs renszerben se tegyenek kárt.
	\item[Állomány- és lemezkezelés (File and Disk Management)] Rendet kell tartania a hosszabb távra megőrzendő állományok között.
	\item[Felhasználói felület (User Interface)] A parancsnyelveket feldolgozó monito utódja, 	fejlettebb változata, melynek segítségével a felhasználó közölni tudja a rendszermaggal kívánságait, illetve annk állapotáról információt szerezhet.
\end{description}

\paragraph{Operációs rendszerek funkciói:}
\begin{description}
\item[Folyamatkezelés]
A folyamat egy végrehajtás alatt álló program. Hogy feladatát ellássa erőforrásokra van szüksége (processzor idő, memória, állományok I/O berendezések).
Az operációs rendszer feladata:
\begin{itemize}
	\item Folyamatok létrehozása és törlése
	\item Folyamatok felfüggesztése és újraindítása
	\item Eszközök biztosítása a folyamatok kommunikációjához és szinkronizációjához.
\end{itemize}
\item[Memória (főtár) kezelés]
Bájtokból álló tömbnek tekinthető, amelyet a CPU és az I/O közösen használ. Tartalma törlődik rendszerkikapcsoláskor és rendszerhibáknál.
Az operációs rendszer feladata:
\begin{itemize}
	\item Nyilvántartani, hogy az operatív memória melyik részét ki (mi) használja.
	\item Eldönteni melyik folyamatot kell betölteni, ha memória felszabadul.
	\item Szükség szerint allokálni és felszabadítani a memória területeket a szükségleteknek megfelelően.
\end{itemize}
\item[Másodlagos tárkezelés]
Nem törlődik, és elég nagy hogy minden programot tároljon. A merevlemez a legelterjedtebb
formája. Az operációs rendszer feladata:
\begin{itemize}
	\item Szabadhely kezelés.
	\item Tárhozzárendelés.
	\item Lemez elosztás.
\end{itemize}
\item[I/O rendszerkezelés]
\begin{itemize}
	\item Puffer rendszer.
	\item Általános készülék meghajtó (device driver) interface.
	\item Speciális készülék meghajtó programok.
\end{itemize}
\item[Fájlkezelés]
Egy fájl kapcsolódó információk együttese, amelyet a létrehozója definiál. Általában program és adatfájlokról beszélünk.
Az operációs rendszer feladata:
\begin{itemize}
	\item Fájlok és könyvtárak létrehozás és törlése.
	\item Fájlokkal és könyvtárakkal történő alapmanipuláció.
	\item Fájlok leképezése a másodlagos tárra, valamilyen nem törlődő, stabil adathordozóra.
\end{itemize}
\item[Védelmi rendszer]
Olyan mechanizmus, mely az erőforrásokhoz való hozzá férést felügyeli. Az operációs rendszer feladata:
\begin{itemize}
	\item Különbséget tenni jogos (authorizált) és jogtalan használat között.
	\item Specifikálni az alkalmazandó kontrolt.
	\item Korlátozó eszközöket szolgáltatni.
\end{itemize}
\item[Hálózat elérés támogatása]
Az elosztott rendszer processzorok adat és vezérlő vonallal összekapcsolt együttese, ahol a memória és az óra nem közös. Adat- és vezérlővonal segítségével történik a kommunikáció. Az elosztott rendszer a felhasználóknak különböző osztott erőforrások elérését teszi lehetővé, mely lehetővé teszi:
\begin{itemize}
	\item a számítások felgyorsítását,
	\item a jobb adatelérhetőséget,
	\item a nagyobb megbízhatóságot.
\end{itemize}
\item[Parancs interpreter alrendszer]
Az operációs rendszernek sok parancsot vezérlő utasítás formájában lehet megadni. Vezérlő utasítások minden területhez tartoznak (folyamatok, I/O kezelés...). Az operációs rendszernek azt a programját, amelyik a vezérlő utasítást beolvassa és interpretálja a rendszertől függően más és más módon nevezhetik:
\begin{itemize}
	\item Vezérlő kártya interpreter.
	\item Parancs sor interpreter (command line).
	\item Héj (burok, shell)
\end{itemize}
\end{description}

\subsubsection{A rendszeradminisztráció, fejlesztői és alkalmazói támogatás eszközei.}
\paragraph{Rendszeradminisztráció} 
Magának az operációs rendszernek a működtetésével kapcsolatos funkciók. Ezek közvetlenül semmire sem használhatók, csak a hardverlehetőségek kibővítését célozzák, illetve a hardver kezelését teszik kényelmesebbé. A rendszeradminisztráción belül a következő \emph{összetett funkciókat} jelölhetjük ki:
\begin{enumdescript}
	\item[processzorütemezés:] a CPU-idő szétosztása a rendszer- és a felhasználói feladatok
	(taszkok, folyamatok) között;
	\item[megszakításkezelés:] a hardver-szoftver megszakításkérések elemzése, állapotmentés,
	a kezelőprogram hívása;
	\item[szinkronizálás:] az események és az erőforrásigények várakozási sorokba állítása;
	\item[folyamatvezérlés:] a programok indítása és a programok közötti kapcsolatok
	szervezése;
	\item[tárkezelés:] a főtár, -- mint kiemelten kezelt erőforrás, -- elosztása;
	\item[perifériakezelés:] a bemeneti/kimeneti (B/K ill. I/O) igények sorba állítása és
	kielégítése;
	\item[adatkezelés:] az adatállományokon végzett műveletek segítése (létrehozás, nyitás,
	zárás, írás, olvasás stb.);
	\item[működés-nyilvántartás:] a hardver hibastatisztika vezetése és a számlaadatok
	feljegyzése;
	\item[operátori interfész:] a kapcsolattartás az üzemeltetővel.
\end{enumdescript}
A konkrét operációs rendszerek a funkciókat másképpen oszthatják fel. Így például az IBM OS operációs rendszerek változataiban négy fő funkciót szoktak megkülönböztetni:
\begin{enumerate}
	\item a munkakezelést,
	\item a taszkkezelést,
	\item az adatkezelést és
	\item a rendszerstatisztikát.
\end{enumerate}
A rendszeradminisztrációs funkciókat a \textbf{rendszermag} valósítja meg, amelynek a szolgáltatásait a már említett rendszerhívásokkal érhetjük el.

\paragraph{Programfejlesztési támogatás} fő funkciói:
\begin{enumdescript}
	\item[rendszerhívások:] a programokból alacsony szintű operációsrendszeri funkciók
	aktivizálására,
	\item[szövegszerkesztők:] a programok és dokumentációk írására,
	\item[programnyelvi eszközök:] fordítóprogramok és interpreterek (értelmezők) a nyelvek
	fordítására vagy értelmezésére,
	\item[szerkesztő- és betöltő-programok:] a programmodulok összefűzésére illetve tárba
	töltésére (végcímzés),
	\item[programkönyvtári funkciók:] a különböző programkönyvtárak használatára,
	\item[nyomkövetési rendszer:] a programok belövésére.
\end{enumdescript}

\paragraph{Alkalmazói támogatás}
Az alkalmazói támogatás funkciói a számítógépes rendszer több szintjén valósulnak meg, és az alábbi fő funkciókra bonthatók:
\begin{enumdescript}
	\item[operátori parancsnyelvi rendszer:] a számítógép géptermi üzemvitelének
	támogatására;
	\item[munkavezérlő parancsnyelvi rendszer:] a számítógép alkalmazói szintű
	igénybevételének megfogalmazására;
	\item[rendszerszolgáltatások:] az operációs rendszer magjával közvetlenül meg nem oldható
	rendszerfeladatokra;
	\item[segéd-programkészlet:] rutinfeladatok megoldására;
	\item[alkalmazói programkészlet:] az alkalmazásfüggő feladatok megoldására
\end{enumdescript}
%----------------------------------------------------------------------------
\subsection{Magas szintű Programozási nyelvek}
%----------------------------------------------------------------------------
\subsubsection{Adattípus, konstans, változó, kifejezés.}

\subsubsection{Paraméterkiértékelés, paraméterátadás.}

\subsubsection{Hatáskör, névterek, élettartam. }

\subsubsection{Fordítási egységek, kivételkezelés.}

%----------------------------------------------------------------------------
\section{Magas Sintű programozási nyelvek 2}
%----------------------------------------------------------------------------
\subsection{Speciális programnyelvi eszközök.}
Nem érdemel külön fejezetet

\subsection{Az objektumorientált programozás eszközei és jelentősége.}
Az objektumorientált (OO) paradigma középpontjában a programozási nyelvek absztrakciós szintjének növelése
áll. Ezáltal egyszerűbbé, könnyebbé válik a modellezés, a valós világ jobban leírható, a valós problémák
hatékonyabban oldhatók meg. Az OO szemlélet szerint az adatmodell és a funkcionális modell egymástól
elválaszthatatlan, külön nem kezelhető. A valós világot egyetlen modellel kell leírni és ebben kell kezelni a
statikus (adat) és a dinamikus (viselkedési) jellemzőket. Ez az egységbezárás elve.

\subsection{Funkcionális és logikai programozás.}

%----------------------------------------------------------------------------
\section{Adatszerkezetek és algoritmusok}
%----------------------------------------------------------------------------
\subsection{Adatszerkezetek reprezentációja.}

\subsection{Műveletek adatszerkezetekkel.}

\subsection{Adatszerkezetek osztályozása és jellemzésük. }

\subsection{Szekvenciális adatszerkezetek: sor, verem, lista, sztring.}

\subsection{Egyszerű és összetett állományszerkezetek.}

%----------------------------------------------------------------------------
\section{Adatbázisrendszerek}
%----------------------------------------------------------------------------
\subsection{Relációs, ER és objektumorientált modellek jellemzése.}

\subsection{Adatbázisrendszer.}

\subsection{Funkcionális függés.}

\subsection{Relációalgebra és relációkalkulus. }

\subsection{Az SQL.}

%----------------------------------------------------------------------------
\section{Hálózati architektúrák}
%----------------------------------------------------------------------------
\subsection{Az ISO OSI hivatkozási modell.}

\subsection{Ethernet szabványok.}

\subsection{A hálózati réteg forgalomirányító mechanizmusai. }

\subsection{Az internet hálózati protokollok, legfontosabb szabványok és szolgáltatások.}

%----------------------------------------------------------------------------
\subsection{Fizika 1}
%----------------------------------------------------------------------------
\subsubsection{Fizikai fogalmak, mennyiségek.}

\subsubsection{Impulzus, impulzusmomentum.}

\subsubsection{Newton törvényei.}

\subsubsection{Munkatétel.}

\subsubsection{Az I. és II. főtétel.}

\subsubsection{A kinetikus gázmodell.}

%----------------------------------------------------------------------------
\section{Fizika 2}
%----------------------------------------------------------------------------
\subsection{Elektromos alapfogalmak és alapjelenségek.}

\subsection{Ohm-törvény.}

\subsection{A mágneses tér tulajdonságai. }

\subsection{Elektromágneses hullámok. }

\subsection{A Bohr-féle atommodell.}

\subsection{A radioaktív sugárzás alapvető tulajdonságai.}

%----------------------------------------------------------------------------
\section{Elektronika 1, 2}
%----------------------------------------------------------------------------
\subsection{Passzív áramköri elemek tulajdonságai, RC és RLC hálózatok.}

\subsection{Diszkrét félvezető eszközök, aktív áramköri elemek, alapkapcsolások.}

\subsection{Integrált műveleti erősítők.}

\subsection{Tápegységek.}

\subsection{Mérőműszerek.}

%----------------------------------------------------------------------------
\section{Digitális Technika}
%----------------------------------------------------------------------------
\subsection{Logikai függvények kapcsolástechnikai megvalósítása.}

\subsection{Digitális áramköri családok jellemzői(TTL, CMOS, NMOS).}

\subsection{Különböző áramköri családok csatlakoztatása.}

\subsection{Kombinációs és szekvenciális hálózatok. A/D és D/A átalakítók.}


\section[Infokommunikációs hálózatok]{Infokommunikációs hálózatok specializáció}
%----------------------------------------------------------------------------
\subsection{Távközlő hálózatok}
%----------------------------------------------------------------------------
\subsubsection{Fizikai jelátviteli közegek.}

\subsubsection{Forráskódolás, csatornakódolás és moduláció.}

\subsubsection{Csatornafelosztás és multiplexelési technikák.}

\subsubsection{Vezetékes és a mobil távközlő hálózatok.}

\subsubsection{Műholdas kommunikáció és helymeghatározás.}

%----------------------------------------------------------------------------
\subsection{Hálózatok hatékonyságanalízise}
%----------------------------------------------------------------------------
\subsubsection{Markov-láncok, születési-kihalási folyamatok.}

\subsubsection{A legalapvetőbb sorbanállási rendszerek vizsgálata.}

\subsubsection{A rendszerjellemzők meghatározásának módszerei, meghatározásuk számítógépes támogatása.}

%----------------------------------------------------------------------------
\subsection{Adatbiztonság}
%----------------------------------------------------------------------------
\subsubsection{Fizikai, ügyviteli és algoritmusos adatvédelem, az informatikai biztonság szabályozása.}

\subsubsection{Kriptográfiai alapfogalmak.}

\subsubsection{Klasszikus titkosító módszerek.}

\subsubsection{Digitális aláírás, a DSA protokoll.}

%----------------------------------------------------------------------------
\subsection{A RIP protokoll működése és paramétereinek beállítása (konfigurációja).}
%----------------------------------------------------------------------------

%----------------------------------------------------------------------------
\section{Bevezetés a Cisco eszközök programozásába 1}
%----------------------------------------------------------------------------
\subsection{A forgalomszűrés, forgalomszabályozás (Trafficfiltering, ACL) céljai és beállítása (konfigurációja) egy választott példa alapján.}

%----------------------------------------------------------------------------
\subsection{Bevezetés a Cisco eszközök programozásába 2}
%----------------------------------------------------------------------------
\subsubsection{A forgalomirányítási táblázatok felépítése, statikus és dinamikus routing összehasonlítása.}

%\section[Mérés és folyamatirányítás]{Mérés és folyamatirányítás specializáció}
%%----------------------------------------------------------------------------
\subsection{}
%----------------------------------------------------------------------------
\subsubsection{Kísérlet-elmélet-szimuláció viszonya.}

\subsubsection{Véletlen folyamatok modellezése: bolyongás és növekedés, a Monte Carlo módszer.}

\subsubsection{A molekuláris dinamikai szimuláció alapjai.}

\subsubsection{Diszkrét dinamikai rendszerek, sejtautomata modellezés.}

%%----------------------------------------------------------------------------
\section{}
%----------------------------------------------------------------------------

%%----------------------------------------------------------------------------
\section{}
%----------------------------------------------------------------------------

%%----------------------------------------------------------------------------
\section{}
%----------------------------------------------------------------------------

%%----------------------------------------------------------------------------
\section{}
%----------------------------------------------------------------------------

%%----------------------------------------------------------------------------
\section{}
%----------------------------------------------------------------------------


%\section[Vállalati információs rendszerek]{Vállalati információs rendszerek specializáció}
%%----------------------------------------------------------------------------
\section{}
%----------------------------------------------------------------------------

%%----------------------------------------------------------------------------
\section{}
%----------------------------------------------------------------------------

%%----------------------------------------------------------------------------
\section{}
%----------------------------------------------------------------------------

%%----------------------------------------------------------------------------
\section{}
%----------------------------------------------------------------------------

%%----------------------------------------------------------------------------
\section{}
%----------------------------------------------------------------------------

%%----------------------------------------------------------------------------
\section{}
%----------------------------------------------------------------------------


\printindex\addcontentsline{toc}{section}{Tárgymutató}
\end{document}
